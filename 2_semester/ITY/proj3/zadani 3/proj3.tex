\documentclass[11pt, a4paper]{article}
\usepackage[utf8]{inputenc}
\usepackage[left=2cm, text={17cm, 24cm}, top=3cm]{geometry}
\usepackage[czech]{babel}
\usepackage[IL2]{fontenc}
\usepackage{tabbing}


\begin{document}
\begin{titlepage}

\thispagestyle{empty}
\begin{center}
\Huge
\textsc{Fakulta informačních technologií}\hspace{0.4em}\\
\textsc{Vysoké učení technické v Brně}\hspace{0.4em}\\
\vspace{\stretch{0.382}}
\LARGE{ Typografie a publikování – 3. projekt \hspace{0.3em}}\\
\Huge {Tabulky a obrázky}\hspace{0.3em}\\
\vspace{\stretch{0.618}}
\end{center}
\Large{31.03.2018 \hfill
Matúš Škuta (xskuta04)}
\end{titlepage}


Pro sázení tabulek můžeme použít buď prostředí tabbing nebo prostředí tabular.
Prostředí tabbing
Při použití tabbing vypadá tabulka následovně:

Toto prostředí se dá také použít pro sázení algoritmů, ovšem vhodnější je použít
prostředí algorithm nebo algorithm2e (viz sekce 3).

Další možností, jak vytvořit tabulku, je použít prostředí tabular. Tabulky pak
budou vypadat takto:

Pokud budeme chtít vysázet algoritmus, můžeme použít prostředí algorithm nebo algorithm2e.
Příklad použití prostředí algorithm2e viz Algoritmus 1.

Kdyby byl problem s cline, zkuste se podívat třeba sem:
http://www.abclinuxu.cz/tex/poradna/show/325037

Pro nápovědu, jak zacházet s prostředím algorithm, můžeme zkusit tuhle stránku:
http://ftp.cstug.cz/pub/tex/CTAN/macros/latex/contrib/algorithms/algorithms.pdf.
Pro algorithm2e zase tuhle:
http://ftp.cstug.cz/pub/tex/CTAN/macros/latex/contrib/algorithm2e/algorithm2e.pdf.

Do našich článků můžeme samozřejmě vkládat obrázky. Pokud je obrázkem fotografie,
můžeme klidně použít bitmapový soubor. Pokud by to ale mělo být nějaké schéma nebo
něco podobného, je dobrým zvykem takovýto obrázek vytvořit vektorově.

Odkazy (nejen ty) na obrázky 1, 2 a 3, na
tabulky 1 a 2 a také na algoritmus 1 jsou udělány pomocí
křížových odkazů. Pak je ovšem potřeba zdrojový soubor přeložit dvakrát.
Vektorové obrázky lze vytvořit i přímo v LATEXu, například pomocí prostředí
picture. Všechny rozměry jsou uváděny v mm.



\end{document}
