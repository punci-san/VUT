\documentclass[11pt, a4paper]{article}
\usepackage[utf8]{inputenc}
\usepackage[left=2cm, text={17cm, 24cm}, top=3cm]{geometry}
\usepackage[czech]{babel}
\usepackage{times}
\usepackage[unicode]{hyperref}


\title{Proj4}
\author{fiederly }
\date{April 2018}

\begin{document}
\begin{titlepage}

\thispagestyle{empty}
\begin{center}
\Huge
\textsc{Vysoké učení technické v Brně}\hspace{0.4em}\\
\huge{\textsc{Fakulta informačních technologií}}\hspace{0.4em}\\
\vspace{\stretch{0.382}}
\LARGE{ Typografie a publikování – 4. projekt \hspace{0.3em}}\\
\Huge {Bibliografické citácie}\hspace{0.3em}\\
\vspace{\stretch{0.618}}
\end{center}
\Large{14. apríl 2018 \hfill
Matúš Škuta}
\end{titlepage}

\section{\LaTeX}

\subsection{Definícia}

\LaTeX je vysoko kvalitný typografický systém určený pre profesionálne a poloprofesionálne sádzanie dokumentov. Bol vyvinutý v roku 1985 Leslie Lamportom.

\subsection{Prečo používať \LaTeX ?}

\LaTeX vyniká obrovskou presnosťou pri vytváraní dokumentov. Je vhodný pre veľké a náročné projekty oproti MS Word viz \cite{WORDvsLATEX}. \LaTeX obsahuje takisto aj libraries pre zložíté matematické vzorce, tabuľky, kreslenie obrázkov a na odkazy, je lepšie použiť \LaTeX viz \cite{Rybicka2003}. Ak chcete zistit či je latex práve pre Vás odporúčam Vám článok viz \cite{IsLatexForMe}. 

\subsection{Práca s \LaTeX om}

Práca so systémom \LaTeX je veľmi ľahká, lebo je veľmi podobná programovaniu viz \cite{Rybicka2003}:
\begin{enumerate}
    \item {Písanie dokumentu}
    \item {Preklad dokumentu}
    \item {Prezeranie dokumentu}
\end{enumerate}
Ak vás toto zaujalo a chceli by ste sa dozvediet viac, môžte si o tom prečítať viz \cite{Rybicka2003}

Najväčšia nevýhoda \LaTeX u spočíva v tom že nie je až tak rozšírený ako MS Word ktorý je skoro na každom počítači. Ale nezúfajte môžme použiť množstvo online editorov vytvorených špeciálne pre \LaTeX.

\subsection{Naučte sa \LaTeX}

Ak vás latex zaujal a chceli by ste sa o nom dozvedieť viac, existuje množstvo knižných materiálov viz \cite{Rybicka2003} alebo \cite{LaTeX2004} a ak nie ste ten typ na knihy nevadí existujú aj online tutoriály. Jeden z online tutoriálov ktoré ja odporúčam je \cite{LearnLATEX} ktorý je napísany veľmi jednoducho a obsahuje množstvo príkladov.
 \LaTeX 
 je neustále rozširovaný o nové funkcie a preto existujú aj nástroje ktoré dokážu vygenerovať z ručne písaného textu zdrojový kód viz \cite{Oksuz2008}


\subsection{Matematické vzorce v \LaTeX e}

\LaTeX je ako stvorený pre matematické vzorce lebo sadzba matematických vzorcov je veľmi ľahká.V \LaTeX e sa vytvárajú vzorce medzi dvoma \$ v ktorých je veľmi ľahké vytvárať zlomky,matematické symboly a znaky viz \cite{Olsak2014}.
Pre príklad vysádzanej rovnice môžme nájsť viz \cite{CazarezCastro2012}




\subsection{Práce zamerané na \LaTeX}
Jedna z bakalárskych prác zameraná na využitie typografického systému \LaTeX na úpravu súborov pomocou tried viz \cite{Hasik2017} a diplomová práca zameraná na editor tabuliek pre systém LaTeX viz \cite{Simovic2015}

	\newpage
	\bibliographystyle{czechiso}
	\renewcommand{\refname}{Literatura}
	\bibliography{proj4}

\end{document}
