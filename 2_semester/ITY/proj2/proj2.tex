\documentclass[11pt, twocolumn, a4paper]{article}
\usepackage[utf8]{inputenc}
\usepackage[left=1.5cm, text={18cm, 25cm}, top=2.5cm]{geometry}
\usepackage[czech]{babel}
\usepackage[IL2]{fontenc}
\usepackage{times}
\usepackage{amsthm}
\usepackage{amsmath}
\usepackage{amsfonts}

\newtheorem{theorem}{Definice}
\newtheorem{sentence}{Věta}

\DeclareMathOperator*{\VDash}{\vdash}


\begin{document}
\begin{titlepage}

\thispagestyle{empty}
\begin{center}
\Huge
\textsc{Fakulta informačních technologií}\hspace{0.4em}\\
\textsc{Vysoké učení technické v Brně}\hspace{0.4em}\\
\vspace{\stretch{0.382}}
\LARGE{ Typografie a publikování – 2. projekt \hspace{0.3em}\\
Sazba dokumentů a matematických výrazů}\hspace{0.3em}\\
\vspace{\stretch{0.618}}
\end{center}
\Large{ 2018 \hfill
Matúš Škuta (xskuta04)}
\end{titlepage}

\setcounter{page}{1}
\section*{Úvod}

V této úloze si vyzkoušíme sazbu titulní strany, matematic\-kých vzorců, prostředí a dalších textových struktur obvyklých
pro technicky zaměřené texty (například rovnice (1) nebo Definice \ref{Definice 1} na straně \pageref{Definice 1}). Rovněž si vyzkoušíme používání
odkazů \verb|\ref| a \verb|\pageref|.

Na titulní straně je využito sázení nadpisu podle optického středu s využitím zlatého řezu. Tento postup byl probírán na přednášce. Dále je použito odřádkování se zadanou relativní velikostí 0.4em a 0.3em.

\section{Matematický text}


Nejprve se podíváme na sázení matematických symbolů
a~výrazů v plynulém textu včetně sazby definic a vět s využitím
balíku \texttt{amsthm}. Rovněž použijeme poznámku pod
čarou s použitím příkazu \verb|\footnote|. Někdy je vhodné
použít konstrukci \verb!${}$!, která říká, že matematický text
nemá být zalomen.
\label{Definice 1}
\begin{theorem}
\emph{Turingův stroj} (TS) je definován jako~šestice
tvaru $M = (Q, \Sigma, \Gamma, \delta, q_0, q_F )$, kde:
\end{theorem}
\begin{itemize}
\item $Q$ \textit{je konečná množina} vnitřních (řídicích) stavů,

\item $\Sigma$ \textit{je konečná množina symbolů nazývaná} vstupní
abeceda, $\Delta \notin \Sigma$,

\item $\Gamma$ \textit{je konečná množina symbolů, $\Sigma \subset \Gamma$, $\Delta \in \Gamma$,
nazývaná} pásková abeceda,

\item $\delta:(Q\backslash\{q_F \})\times \Gamma \rightarrow Q\times(\Gamma\cup\{L, R\})$, \textit{kde $L, R \notin \Gamma$, je parciální} přechodová funkce,

\item  $q_0$ \textit{je} počáteční stav, $q_0 \in Q$ $a$

\item $q_F$ \textit{je} koncový stav, $q_F \in Q$.
\end{itemize}

Symbol $\Delta$ značí tzv. \textit{blank} (prázdný symbol), který se vyskytuje na místech pásky, která nebyla ještě použita
(může ale být na pásku zapsán i později).

\textit{Konfigurace pásky} se skládá z nekonečného řetězce,
který reprezentuje obsah pásky a pozice hlavy na tomto
řetězci. Jedná se o prvek množiny $\{\gamma\Delta^\omega $ $|$ $ \gamma\in\Gamma^\ast\} \times \mathbb{N}$.\footnote{Pro libovolnou abecedu $\Sigma$ je $\Sigma^\omega$ množina všech \textit{nekonečných} řetězců nad $\Sigma$, tj. nekonečných posloupností symbolů ze $\Sigma$. Pro připo\-menutí: $\Sigma^\ast$ je množina všech \textit{konečných} řetězců nad $\Sigma$.} \textit{Konfiguraci pásky} obvykle zapisujeme jako $\Delta xyz\underline{z}x\Delta...$ (podtržení značí pozici hlavy). \textit{Konfigurace stroje} je pak dána stavem řízení a konfigurací pásky. Formálně se jedná o prvek množiny $Q \times \{\gamma\Delta^\omega$ $|$ $ \gamma \in \Gamma^\ast\} \times\mathbb{N}$.

\subsection{Podsekce obsahující větu a odkaz}

\begin{theorem}
\begin{normalfont}Řetězec $w$ {nad abecedou} $\Sigma$ je přijat TS $M$\end{normalfont}
jestliže $M$ při aktivaci z počáteční konfigurace pásky
$\underline{\Delta} w\Delta...$ a počátečního stavu $q_0$ zastaví přechodem do koncového stavu $q_F$, tj. $(q_0, \Delta w\Delta^\omega, 0) \underset{M}{\overset{*}{\vdash}} (q_F , \gamma,n)$ pro nějaké $\gamma \in \Gamma^\ast$ a $n \in \mathbb{N}$.

Množinu $L(M) = \{w$ $|$ $w$ je přijat TS $M\} \subseteq \Sigma^\ast$ nazý\-váme \emph{jazyk přijímaný TS} $M$.
\end{theorem}
Nyní si vyzkoušíme sazbu vět a důkazů opět s použitím
balíku \texttt{amsthm}.
\begin{sentence}
Třída jazyků, které jsou přijímány TS, odpovídá
\emph{rekurzivně vyčíslitelným jazykům.}
\end{sentence}
\begin{proof}
V důkaze vyjdeme z Definice 1 a 2.
\end{proof}

\section{Rovnice a odkazy}
Složitější matematické formulace sázíme mimo plynulý
text. Lze umístit několik výrazů na jeden řádek, ale pak je
třeba tyto vhodně oddělit, například příkazem \verb|\quad|.


$$\begin{normalfont}\sqrt[i]{x^3_i} \quad \text{kde }x_i \text{ je } i\text{-té sudé číslo}\quad  y_i^{2 \cdot y_i}\neq y_i^{y_i^{y_i}} \end{normalfont}$$

V rovnici (1) jsou využity tři typy závorek s různou
explicitně definovanou velikostí.
\begin{eqnarray}
x &=& \left\{\Big(\big[ a+b\big]*c\Big)^d\oplus 1\right\} \\
y &=& \lim_{x\rightarrow\infty}\frac{\sin^2x + \cos^2x}{\frac{1}{\log_{10}x}} \nonumber
\end{eqnarray}

V této větě vidíme, jak vypadá implicitní vysázení limity
$\lim_{n\rightarrow\infty} f(n)$ v normálním odstavci textu. Podobně je to i s dalšími symboly jako $\sum_{i=1}^n 2^i$ či $\bigcup_{A\in \mathcal{B}} A$.
V~případě vzorců $\lim\limits_{n\rightarrow\infty} f(n)$ a $\sum\limits_{i=1}^n 2^i$ jsme si vynutili méně
úspornou sazbu příkazem \verb|\limits|.

\begin{eqnarray}
    \int\limits_a^b f(x)\ \mathrm{d}x &=& -\int_b^a g(x)\  \mathrm{d}x\\
     \overline{\overline{ A \vee B}} &\Leftrightarrow& \overline{\overline{A} \wedge \overline{B}} \quad
\end{eqnarray}

\section{Matice}

Pro sázení matic se velmi často používá prostředí \texttt{array}
a závorky (\verb|\left|, \verb|\right|).

\[ \left( \begin{array}{ccc}
a+b & \widehat{\xi+\omega} & \hat{\pi} \\
\vec{a} & \overleftrightarrow{AC} & \beta \end{array} \right) =1 \Longleftrightarrow \mathbb{Q}=\mathbb{R}\]


$$\mathbf{A}=\left\|\begin{array}{cccc}
a_{11} & a_{12} & \ldots & a_{1n} \\
a_{21} & a_{22} & \ldots & a_{2n} \\
\vdots & \vdots & \ddots & \vdots \\
a_{m1} & a_{m2} & \ldots & a_{mn}
\end{array}\right\| =
\begin{array}{|cc|}
\,t&u\, \\
\,v&w\,
\end{array} = tw\!-\!uv
$$

Prostředí \texttt{array} lze úspěšně využít i jinde.



$$ \binom{n}{k} = \left\{\begin{array}{l l}
    \frac{n!}{k!(n-k)!} & \text{pro } 0 \leq k\ \leq n  \\
0 & \text{pro } k\ < 0 \text{ nebo } k\ > n
\end{array} \right. $$


\section{Závěrem}

V~případě, že budete potřebovat vyjádřit matematickou
konstrukci nebo symbol a nebude se Vám dařit jej nalézt
v~samotném \LaTeX u, doporučuji prostudovat možnosti balíku
maker \AmS-\LaTeX .

\end{document}
